\documentclass[a4paper, 12pt]{article}
\usepackage[utf8]{inputenc}  
\usepackage[brazilian]{babel}  
\usepackage[top=2cm, bottom=2cm, left=2.5cm, right=2.5cm]{geometry}
\begin{document}

\begin{center}
    \underline{Aluno: Luiz Veloso Dos Santos}
\end{center}    

\begin{center} 
    \textbf{1º Trabalho: Sistemas de Numeração e Lógica Binária}
    \end{center}    

    \begin{enumerate}
        \item Converta os seguintes números da base 10 para a base 2:
            \begin{enumerate}
                \item $10,5625_{10} = 1010,1001_{2}$ \ 

                    Cálculo: \

                    $0,5625 \cdot 2 = \textbf{1},125  $\

                    $0,125 \cdot 2 = \textbf{0},25 $\

                    $0,25 \cdot 2 = \textbf{0},5 $\

                    $0,5 \cdot 2 = \textbf{1} $\

                \item $255_{10} = 01111111_2 $
                \item $256_{10} = 10000000_2 $
                \item $1_{10} = 01_2 $
                \item $0_{10} = 0_2 $

            \end{enumerate}

        \item Sabendo que o endereço IP de uma máquina é composto de 4 
            octetos(números de 8 bits) separados por um ponto, quais números
            abaixo podem ser endereços IP válidos? Justifique sua resposta.\

            \textbf{R:} Como o IP é composto de 4 octetos e cada octeto é um número de 8 bits,
            significa que nenhuma parte do IP pode haver um número maior que 255, pois esse é
            o maior valor que pode ser representado em 1 byte(8bits).\
            Sabendo disso os IP's válidos são:

            \begin{enumerate}
                \item \textbf{200.10.5.6}
                \item 
                \item
                \item \textbf{134.132.133.1}
            \end{enumerate}
    \end{enumerate}    
   \end{document}

