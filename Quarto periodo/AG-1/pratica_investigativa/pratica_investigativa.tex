%!TEX TS-program = xelatex
%!TEX encoding = UTF-8 Unicode

\documentclass[a4paper,11pt,fleqn]{article}
\usepackage{graphicx,url}
\usepackage[T1]{fontenc}
\usepackage[brazil]{babel}
\usepackage{a4wide}
\usepackage{booktabs}
\usepackage{amsmath} % For math
\usepackage{fontspec}
\setmainfont{Times New Roman}

\title{\vspace{-4cm}Prática investigativa $-$ Algoritmos em Grafos}
\author{Luiz Junio Veloso Dos Santos $-$ Matricula: 624037}

\begin{document}
\maketitle

\centering{\large{Responda às seguintes perguntas com relação ao artigo\\
“\textit{Multicast Tree Construction and Flooding in Wireless Ad Hoc Networks}”}
}

\begin{enumerate}
    \item{Defina redes \textit{Ad Hoc}.}
    \newline
    \textbf{R:}
        Uma rede \textit{Ad Hoc} consiste em hosts móveis sem fio que formam uma rede
        temporária sem o auxilio da infraestrutura estabelecida ou da administração
        centralizada.

    \item{Quais são os desafios enfrentados pelo roteamento em redes \textit{Ad Hoc}?}
    \newline
    \textbf{R:}
        Os desafios são a topologia da rede, que muda dinamicamente, e
        a largura de banda limitada em uma rede sem fios \textit{Ad Hoc}.

    \item{Qual é a motivação deste artigo?}
    \newline
    \textbf{R:}
        A motivação do artigo é de se criar mecanismos mais eficientes de roteamento
        para \textit{multicast} e \textit{broadcast} em uma rede \textit{Ad Hoc} sem fio.

    \item{Como as redes \textit{Ad Hoc} são modeladas como grafos?}
    \newline
    \textbf{R:}
        Cada terminal móvel é representado como um vértice e
        2 vértices estão conectados por uma aresta quando eles estão dentro
        da área de transmissão de cada um.
        \newline
        Obs: Embora em um ambiente sem fio real, alguns links podem ser unidirecional
        devido a área de transmissão de 2 vértices serem diferentes, isso pode ser mascarado
        usando mecanismos de detecção de link unidirecional, assim pode se assumir
        que todos os links/arestas são bidirecional, se o vértice $v_i$ pode se
        comunicar com $v_j$, então $v_j$ pode também se comunicar com $v_i$.
        \newline
        $N(v)$ é definido como os vértices adjacentes do vértice v.

    \item{Defina o problema MCDS (\textit{Minimum Connected Dominating Set}).}
    \newline
    \textbf{R:}
        O problema MCDS é encontrar o subconjunto conectado minimo S de V
        onde todos os elementos em $V - S$ é adjacente para ao menos um elemento de S,
        dado o grafo $G(V,E)$.

    \item{Explique o funcionamento das duas heurísticas para inundação (\textit{flooding}) propostas.}
    \newline
    \textbf{R:}
        \begin{itemize}
            \item{\textit{Self pruning}:}
            Cada vértice troca uma lista de seus vértices adjacentes com
            seus vizinhos. Cada vértice emite um pacote ``Quem eu sou'' periodicamente
            para informar sua existência para os vértices vizinhos.
            O vértice $v_i$, que deseja encaminhar um pacote, ``pega carona'' na lista
            de vértices adjacentes $N(v_i)$ no pacote ``inundado''.
            O vértice $v_j$, que recebe o pacote, verifica se o conjunto $N(v_j) - V(v_i) - \{v_i\}$
            está vazio. Se estiver, o vértice $v_j$ se abstém de encaminhar o pacote, pois
            ele sabe que todos os seus vértices adjacentes devem ter recebido o pacote
            quando o vértice $v_i$ encaminhou. Caso contrário, o vértice $v_j$ encaminha o pacote.
            \newline
            Para decidir se deve encaminhar o pacote recebido ou não, $v_j$ deve iterar para
            cada $v \in N(v_i)$ descobrindo e removendo v de $N(v_j) - \{v_i\}$. Se todos os elementos
            foram removidos de $N(v_j) - \{v_i\}$, $v_j$ não encaminha o pacote. Senão, $v_j$ encaminha
            o pacote.

        \item{\textit{Dominating pruning}:}
            Enquanto \textit{Self pruning} explora o conhecimento de vizinhos diretamente conectados
            apenas, \textit{Dominant pruning} estende a área de informações da vizinhança em 2
            arestas de distancia $N(N(v))$. Essa informação de vizinhos de ``2 saltos'' pode ser
            obtida trocando as listas de vértices adjacentes com os vizinhos.
            \newline
            No \textit{Dominant pruning}, o vértice de envio seleciona os vértices adjacentes que
            deve retransmitir o pacote para completar a transmissão. Os IDs dos vértices
            adjacentes são reescritos no pacote como uma linha de encaminhamento. Um vértice
            adjacente que é requisitado para retransmitir o pacote novamente determina a
            lista de encaminhamento. Esse processo é iterado até que a transmissão seja
            concluída.

        \end{itemize}
    \item{Escreva sucintamente as contribuições deste artigo.}
    \newline
    \textbf{R:}
        Esse artigo contribuiu ao apresentar 2 novos algoritmos que podem
        melhorar a transmissão de pacotes em redes Ad Hoc sem fio, o que irá
        contribuir com uma maior velocidade de troca de dados nos ambientes que
        fazem uso desse tipo de rede, como campos de combate e áreas de resgate cujo
        tempo é um fator crucial.

\end{enumerate}

\end{document}
