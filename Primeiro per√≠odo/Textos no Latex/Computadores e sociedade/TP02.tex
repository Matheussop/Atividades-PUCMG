% MODELO DE LATEX PARA TRABALHOS ACADÊMICOS
%% INSTRUÇÕES GERAIS:
%%    1. TODO O TEXTO NA FRENTE DO SIMBOLO '%' É COMENTÁRIO, ISTO É, ELE NÃO FAZ DIFERENÇA NO RESULTADO FINAL 
%%    2. NESTE MODELO, VOCÊS SÓ PRECISAM EDITAR DAS LINHAS 114 A 132 (INFORMAÇÕES DE CAPA) E DAS LINHAS 188 EM DIANTE (CORPO DO TRABALHO). O RESTO SÃO CONFIGURAÇÕES DE FORMATAÇÃO QUE PROVAVELMENTE NÃO SERÁ PRECISO MODIFICAR.
%%    3. MAIS INSTRUÇÕES DETALHADAS PODERÃO SER ENCONTRADAS NA PÁGINA profhelioh.wordpress.com. DÚVIDAS: heliohenrique@ufpr.br OU heliohenrique3@gmail.com

% INFORMAÇÕES DA FONTE:
%% abtex2-modelo-relatorio-tecnico.tex, v-1.7.1 laurocesar
%% Copyright 2012-2013 by abnTeX2 group at http://abntex2.googlecode.com/ 
%%
%% This work may be distributed and/or modified under the
%% conditions of the LaTeX Project Public License, either version 1.3
%% of this license or (at your option) any later version.
%% The latest version of this license is in
%%   http://www.latex-project.org/lppl.txt
%% and version 1.3 or later is part of all distributions of LaTeX
%% version 2005/12/01 or later.
%%
%% This work has the LPPL maintenance status `maintained'.
%% 
%% The Current Maintainer of this work is the abnTeX2 team, led
%% by Lauro César Araujo. Further information are available on 
%% http://abntex2.googlecode.com/
%%
%% This work consists of the files abntex2-modelo-relatorio-tecnico.tex,
%% abntex2-modelo-include-comandos and abntex2-modelo-references.bib
%%
% ------------------------------------------------------------------------
% ------------------------------------------------------------------------
% abnTeX2: Modelo de Relatório Técnico/Acadêmico em conformidade com 
% ABNT NBR 10719:2011 Informação e documentação - Relatório técnico e/ou
% científico - Apresentação
% ------------------------------------------------------------------------ 
% ------------------------------------------------------------------------

\documentclass[
% -- opções da classe memoir --
12pt,				% tamanho da fonte
% openright,			% capítulos começam em pág ímpar (insere página vazia caso preciso)
oneside,			% para impressão somente frente. Oposto a twoside (frente e verso)
a4paper,			% tamanho do papel. 
% -- opções da classe abntex2 --
%chapter=TITLE,		% títulos de capítulos convertidos em letras maiúsculas
%section=TITLE,		% títulos de seções convertidos em letras maiúsculas
%subsection=TITLE,	% títulos de subseções convertidos em letras maiúsculas
%subsubsection=TITLE,% títulos de subsubseções convertidos em letras maiúsculas
% -- opções do pacote babel --
english,			% idioma adicional para hifenização
french,				% idioma adicional para hifenização
spanish,			% idioma adicional para hifenização
brazil,				% o último idioma é o principal do documento
]{abntex2}


% ---
% PACOTES
% ---

% ---
% Pacotes fundamentais 
% ---
\usepackage{cmap}				% Mapear caracteres especiais no PDF
\usepackage{lmodern}			% Usa a fonte Latin Modern
\usepackage[T1]{fontenc}		% Selecao de codigos de fonte.
\usepackage[utf8]{inputenc}		% Codificacao do documento (conversão automática dos acentos)
\usepackage{indentfirst}		% Indenta o primeiro parágrafo de cada seção.
\usepackage{color}				% Controle das cores
\usepackage{graphicx}			% Inclusão de gráficos
% ---

% ---
% Pacotes adicionais, usados no anexo do modelo de folha de identificação
% ---
\usepackage{multicol}
\usepackage{multirow}
% ---

% ---
% Pacotes adicionais, usados apenas no âmbito do Modelo Canônico do abnteX2
% ---
\usepackage{lipsum}				% para geração de dummy text
% ---

% ---
% Pacotes de citações
% ---
\usepackage[brazilian,hyperpageref]{backref}	 % Paginas com as citações na bibl
\usepackage[alf]{abntex2cite}	% Citações padrão ABNT

% --- 
% CONFIGURAÇÕES DE PACOTES
% --- 

% ---
% Configurações do pacote backref
% Usado sem a opção hyperpageref de backref
\renewcommand{\backrefpagesname}{Citado na(s) página(s):~}
% Texto padrão antes do número das páginas
\renewcommand{\backref}{}
% Define os textos da citação
\renewcommand*{\backrefalt}[4]{
    \ifcase #1 %
        Nenhuma citação no texto.%
        \or
        Citado na página #2.%
    \else
        Citado #1 vezes nas páginas #2.%
\fi}%
% ---

% ---
% Informações de dados para CAPA e FOLHA DE ROSTO
% ---
\titulo{Trabalho de Informática}
\autor{Fulano da Silva}
\local{Brasil}
\data{20 de novembro de 2014}
\instituicao{%
    Universidade Federal do Paraná
    \par
    Setor Palotina
    \par
Engenharia de Aquicultura}
\tipotrabalho{Relatório técnico}
% O preambulo deve conter o tipo do trabalho, o objetivo, 
% o nome da instituição e a área de concentração 
\preambulo{Modelo canônico de Relatório Técnico e/ou Científico em conformidade
com as normas ABNT apresentado à comunidade de usuários \LaTeX.}
% ---

% ---
% Configurações de aparência do PDF final

% alterando o aspecto da cor azul
\definecolor{blue}{RGB}{41,5,195}


% informações do PDF
\makeatletter
\hypersetup{
    %pagebackref=true,
    pdftitle={\@title}, 
    pdfauthor={\@author},
    pdfsubject={\imprimirpreambulo},
    pdfcreator={LaTeX with abnTeX2},
    pdfkeywords={abnt}{latex}{abntex}{abntex2}{relatório técnico}, 
    colorlinks=true,       		% false: boxed links; true: colored links
    linkcolor=blue,          	% color of internal links
    citecolor=blue,        		% color of links to bibliography
    filecolor=magenta,      		% color of file links
    urlcolor=blue,
    bookmarksdepth=4
}
\makeatother
% --- 

% --- 
% Espaçamentos entre linhas e parágrafos 
% --- 

% O tamanho do parágrafo é dado por:
\setlength{\parindent}{1.3cm}

% Controle do espaçamento entre um parágrafo e outro:
\setlength{\parskip}{0.2cm}  % tente também \onelineskip

% ---
% compila o indice
% ---
\makeindex
% ---

% ----
% Início do documento
% ----
\begin{document}

% Retira espaço extra obsoleto entre as frases.
\frenchspacing 

% ----------------------------------------------------------
% ELEMENTOS PRÉ-TEXTUAIS
% ----------------------------------------------------------
% \pretextual

% ---
% Capa
% ---
\imprimircapa
% ---

% ---
% Folha de rosto
% (o * indica que haverá a ficha bibliográfica)
% ---
\imprimirfolhaderosto*
% ---


% ---
% Agradecimentos
% ---
\begin{agradecimentos}
    O agradecimento principal é direcionado a Youssef Cherem, autor do
    \nameref{formulado-identificacao} (\autopageref{formulado-identificacao}).

    Os agradecimentos especiais são direcionados ao Centro de Pesquisa em
    Arquitetura da Informação\footnote{\url{http://www.cpai.unb.br/}} da Universidade de
    Brasília (CPAI), ao grupo de usuários
    \emph{latex-br}\footnote{\url{http://groups.google.com/group/latex-br}} e aos
    novos voluntários do grupo
    \emph{\abnTeX}\footnote{\url{http://groups.google.com/group/abntex2} e
    \url{http://abntex2.googlecode.com/}}~que contribuíram e que ainda
    contribuirão para a evolução do abn\TeX.

\end{agradecimentos}
% ---

% ---
% RESUMO
% ---

% resumo na língua vernácula (obrigatório)
\begin{resumo} %% AQUI COMEÇA A PÁGINA DE RESUMO
    Segundo a \citeonline{NBR6028:2003}, o resumo deve ressaltar o
    objetivo, o método, os resultados e as conclusões do documento. A ordem e a extensão
    destes itens dependem do tipo de resumo (informativo ou indicativo) e do
    tratamento que cada item recebe no documento original. O resumo deve ser
    precedido da referência do documento, com exceção do resumo inserido no
    próprio documento. (\ldots) As palavras-chave devem figurar logo abaixo do
    resumo, antecedidas da expressão Palavras-chave:, separadas entre si por
    ponto e finalizadas também por ponto. Bla bla bla bla bla \cite{fulano} %% EXEMPLO DE CITAÇÃO (vá em abntex2-modelo-references.bib)

    \vspace{\onelineskip}

    \noindent
    \textbf{Palavras-chaves}: latex. abntex. editoração de texto.
\end{resumo} %AQUI TERMINA A PÁGINA DE RESUMO
% ---

% ---
% inserir lista de ilustrações
% ---

\listoffigures* %% o * indica que não será incluso no sumário
\cleardoublepage %% Pula página
% ---

% ---
% inserir lista de tabelas
% ---

\listoftables*
\cleardoublepage
% ---

% ---
% inserir lista de abreviaturas e siglas
% ---
\begin{siglas}
\item[Fig.] Area of the $i^{th}$ component
\item[456] Isto é um número
\item[123] Isto é outro número
\item[lauro cesar] este é o meu nome
\end{siglas}
% ---

% ---
% inserir lista de símbolos
% ---
\begin{simbolos}
\item[$ \Gamma $] Letra grega Gama
\item[$ \Lambda $] Lambda
\item[$ \zeta $] Letra grega minúscula zeta
\item[$ \in $] Pertence
\end{simbolos}
% ---

% ---
% inserir o sumario
% ---

\tableofcontents*

% ---

% ----------------------------------------------------------
% ELEMENTOS TEXTUAIS  (necessário para incluir número nas páginas)
% ----------------------------------------------------------
\textual


% ----------------------------------------------------------
% Introdução
% ----------------------------------------------------------
\chapter{Introdução} %% NOVO CAPÍTULO (REPARE QUE ELE AUTOMATICAMENTE JÁ COLOCA O NÚMERO DO CAPÍTULO E JÁ ADICIONA NO SUMÁRIO)

Aqui você começa a escrever seu trabalho \cite{fulano}. Este documento e seu código-fonte são exemplos de referência de uso da classe
\textsf{abntex2} e do pacote \textsf{abntex2cite}. O documento 
exemplifica a elaboração de relatórios técnicos e/ou científicos produzidos
conforme a ABNT NBR 10719:2011 \textit{Informação e documentação - Relatório
técnico e/ou científico - Apresentação}.

\begin{table}[hbt] %% EXEMPLO DE TABELA FEITA POR MEIO DO http://www.tablesgenerator.com/
    \begin{center}
        \caption{Nome da tabela.} %% LEGENDA (REPARE QUE ELE JÁ COLOCA A NUMERAÇÃO AUTOMATICAMENTE E JÁ ADICIONA À LISTA DE TABELAS
        \begin{tabular}{|l|l|lll}
            \cline{1-2}
            Produto & Valor   &  &  &  \\ \cline{1-2}\cline{1-2}
            sdadasd & asdasd  &  &  &  \\ \cline{1-2}
            asdasd  & asasdas &  &  &  \\ \cline{1-2}
            dsad    & asdas   &  &  &  \\ \cline{1-2}
        \end{tabular}
    \end{center}
\end{table}

A expressão \textit{\textbf{``Modelo canônico''}} é utilizada para indicar que \abnTeX\ não é
modelo específico de nenhuma universidade ou instituição, mas que implementa tão
somente os requisitos das normas da ABNT. Uma lista completa das normas
observadas pelo \abnTeX\ é apresentada em \citeonline{abntex2classe}. %% USE \citeonline PARA CITAR NO MEIO DA FRASE

\begin{figure} [hbt] 
    %% hbt SIGNIFICA QUE ELE PRIMEIRO VAI TENTAR COLOCAR A IMAGEM NESTE LUGAR (h de "here"). SENÃO DER, ELE TENTA COLOCAR MAIS PRA BAIXO (b de "bottom"). SENÃO ELE COLOCA MAIS PARA CIMA (t de "top").
    \label{figura1} 
    %% LABEL SERVE PARA VOCÊ REFERENCIAR A FIGURA NO MEIO DO TEXTO (VEJA LINHA 330: \ref{figura1}). ASSIM VOCÊ NÃO PERDE A REFERÊNCIA QUANDO MUDA A FIGURA DE LUGAR
    \caption{Exemplo de figura.}
    \includegraphics[width=0.95\textwidth]{nemo.jpeg} %% PARA COLOCAR O ARQUIVO DA IMAGEM NO SHARELATEX, CLIQUE NO ÍCONE QUE PARECE UMA FLECHINHA PARA CIMA (ATUALIZAR), CLIQUE EM UPLOAD E PROCURE A IMAGEM EM SEU COMPUTADOR.
\end{figure}


Sinta-se convidado a participar do projeto \abnTeX! Acesse o site do projeto em
\url{http://abntex2.googlecode.com/} (Figura \ref{figura1}). Também fique livre para conhecer,
estudar, alterar e redistribuir o trabalho do \abnTeX, desde que os arquivos
modificados tenham seus nomes alterados e que os créditos sejam dados aos
autores originais, nos termos da ``The \LaTeX\ Project Public
License''\footnote{\url{http://www.latex-project.org/lppl.txt}}.

Encorajamos que sejam realizadas customizações específicas deste exemplo para
universidades e outras instituições --- como capas, folhas de rosto, etc.
Porém, recomendamos que ao invés de se alterar diretamente os arquivos do
\abnTeX, distribua-se arquivos com as respectivas customizações.
Isso permite que futuras versões do \abnTeX~não se tornem automaticamente
incompatíveis com as customizações promovidas. Consulte
\citeonline{abntex2-wiki-como-customizar} par mais informações.

Este documento deve ser utilizado como complemento dos manuais do \abnTeX\ 
\cite{abntex2classe,abntex2cite,abntex2cite-alf} e da classe \textsf{memoir}
\cite{memoir}. 

Equipe \abnTeX 

Lauro César Araujo


% ---
% Capitulo de revisão de literatura
% ---
\chapter{Lorem ipsum dolor sit amet}

% --- Seção dentro do capítulo
\section{Aliquam vestibulum fringilla lorem}
% ---

\lipsum[1]  %% COMANDO QUE COLOCA TEXTO AUTOMÁTICO, SUBSTITUA POR SEU PRÓPRIO TEXTO

\lipsum[2-3]



% ---
% Conclusão
% ---
\chapter*[Conclusão]{Conclusão}
\addcontentsline{toc}{chapter}{Conclusão}

\lipsum[31-33]


% ----------------------------------------------------------
% ELEMENTOS PÓS-TEXTUAIS
% ----------------------------------------------------------
\postextual


% ----------------------------------------------------------
% Referências bibliográficas
% ----------------------------------------------------------
\bibliography{abntex2-modelo-references} %% REFERENCIA AO ARQUIVO abntex2-modelo-references.bib

% ----------------------------------------------------------
% Glossário
% ----------------------------------------------------------
%
% Consulte o manual da classe abntex2 para orientações sobre o glossário.
%
%\glossary

% ----------------------------------------------------------
% Apêndices
% ----------------------------------------------------------

% ---
% Inicia os apêndices
% ---
\begin{apendicesenv}

    % Imprime uma página indicando o início dos apêndices
    \partapendices

    % ----------------------------------------------------------
    \chapter{Quisque libero justo}
    % ----------------------------------------------------------

    \lipsum[50]

    % ----------------------------------------------------------
    \chapter{Nullam elementum urna vel imperdiet sodales elit ipsum pharetra ligula
    ac pretium ante justo a nulla curabitur tristique arcu eu metus}
    % ----------------------------------------------------------
    \lipsum[55-57]

\end{apendicesenv}
% ---


% ----------------------------------------------------------
% Anexos
% ----------------------------------------------------------

% ---
% Inicia os anexos
% ---
\begin{anexosenv}

    % Imprime uma página indicando o início dos anexos
    \partanexos

    % ---
    \chapter{Morbi ultrices rutrum lorem.}
    % ---
    \lipsum[30]

    % ---
    \chapter{Cras non urna sed feugiat cum sociis natoque penatibus et magnis dis
    parturient montes nascetur ridiculus mus}
    % ---

    \lipsum[31]

    % ---
    \chapter{Fusce facilisis lacinia dui}
    % ---

    \lipsum[32]

\end{anexosenv}

%---------------------------------------------------------------------
% INDICE REMISSIVO
%---------------------------------------------------------------------

\printindex

%---------------------------------------------------------------------
% Formulário de Identificação (opcional)
%---------------------------------------------------------------------
\chapter*[Formulário de Identificação]{Formulário de Identificação}
\addcontentsline{toc}{chapter}{Exemplo de Formulário de Identificação}
\label{formulado-identificacao}

Exemplo de Formulário de Identificação, compatível com o Anexo A (informativo)
da ABNT NBR 10719:2011. Este formulário não é um anexo. Conforme definido na
norma, ele é o último elemento pós-textual e opcional do relatório.

\bigskip

\begin{tabular}{|p{9cm}|p{5cm}|} %% EXEMPLO DE TABELA MAIS COMPLEXA
    \hline
    \multicolumn{2}{|c|}{\textbf{\large Dados do Relatório Técnico e/ou científico}}\\
    \hline
    \multirow{4}{10cm}[24pt]{Título e subtítulo}& Classificação de segurança\\
                                                & \\
                                                \cline{2-2}
                                                & No.\\
                                                & \\

                                                \hline
    Tipo de relatório & Data\\
    \hline
    Título do projeto/programa/plano & No.\\
    \hline
    \multicolumn{2}{|l|}{Autor(es)} \\
    \hline
    \multicolumn{2}{|l|}{Instituição executora e endereço completo} \\
    \hline
    \multicolumn{2}{|l|}{Instituição patrocinadora e endereço completo} \\
    \hline
    \multicolumn{2}{|l|}{Resumo}\\[3cm]
    \hline
    \multicolumn{2}{|l|}{Palavras-chave/descritores}\\
    \hline
    \multicolumn{2}{|l|}{
    Edição \hfill No. de páginas \hfill No. do volume \hfill Nº de classificação \phantom{XXXX}} \\
    \hline
    \multicolumn{2}{|l|}{
    ISSN \hfill \hfill Tiragem \hfill Preço \phantom{XXXXXXXX}} \\
    \hline
    \multicolumn{2}{|l|}{Distribuidor} \\
    \hline
    \multicolumn{2}{|l|}{Observações/notas}\\[3cm]
    \hline
\end{tabular}

\end{document}




