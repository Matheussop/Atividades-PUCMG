\documentclass[12pt]{article}

\usepackage{sbc-template}
\usepackage{graphicx,url}
\usepackage[utf8]{inputenc}
\usepackage[brazil]{babel}  

\sloppy

\title{Resumo de Somátorios}

\author{Luiz Junio Veloso Dos Santos}

\address{Departamento de Ciência da Computação\\ Pontificia Universidade Catolica
  (PUC-MG)\\
  Caixa Postal 1.686 -- 30535-901 -- Belo Horizonte -- MG -- Brasil
  \email{ljvsantos@sga.pucminas.br}
}

\begin{document} 

\maketitle

\begin{enumerate}
    \item O que é somátorio?
    \\
    Somátorio significa a soma de termos, sendo geralmente associada ao operador mátematico da soma de termos
    de uma sequencia. Usualmente é comum o uso da letra grega sigma maiúscula ($\Sigma$) para representar de forma mais clara uma
    soma de n termos de uma sequencia.\\
    Exemplo:
    \begin{enumerate}
        \item Convencional:
            $$S = 1 + 2 + 3 + 4 + 5 + 6 + 7 + 8 + 9 + 10$$
        \item Representação com Sigma:
            $$S = \sum_{N = 1}^{10} N $$
    \end{enumerate}
    ~\\ Exemplo 2: Soma dos números pares até 20
    \begin{enumerate}
        \item Convencional:
            $$ S = 0 + 2 + 4 + 6 + 8 + 10 + 12 + 14 + 16 + 18 + 20 $$
        \item Representação com Sigma:
            $$ 
            S = \sum_{N = 0}^{10} N * 2 = 0*2 + 1*2 + 2*2 + 3*2 + 4*2 + 5*2 + 6*2 + 7*2 + 8*2 + 9*2 + 10*2
            $$
        \end{enumerate}
            Lê-se um somátorio $\sum\limits_{i = 1}^{n} X_i$   como: somatório de X índice i, com i
            variando de 1 até n, onde:\\
            \textbf{n,} é a ordem da última parcela do Somátorio;
            \\
            \textbf{i = 1,} é a ordem da primeira parcela da soma;
            \\
            \textbf{i,} é o índice que está indexando os valores da variável X (outras letras tambem podem
            ser usadas).
            \\
            No caso do exemplo 2b, podemos ler como: somatório de N vezes 2, com N
            variando de 0 a 10.
            \vspace{4mm} 
        \item Propriedades dos Somátorios:
            \begin{enumerate}
                \item Propriedade Aditiva:
                    $$\sum_{i=m}^{n}(a_i + b_i) = \sum_{i=m}^{n} a_i  + \sum_{i=m}^{n} b_i$$
            \end{enumerate}

\end{enumerate}


\bibliographystyle{sbc}
\bibliography{sbc-template}

\end{document}
